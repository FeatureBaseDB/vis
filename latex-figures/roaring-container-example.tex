\documentclass{standalone}
\usepackage[table]{xcolor} % for table cell colors
\renewcommand{\familydefault}{\sfdefault} % for sans font
\begin{document}

\definecolor{pilosa-lightblue}{HTML}{E4EFF4}
\definecolor{pilosa-blue}{HTML}{3C5F8D}
\definecolor{pilosa-green}{HTML}{1DB598}
\definecolor{pilosa-red}{HTML}{ff2B2B}

% https://tex.stackexchange.com/questions/62573/using-underbrace-with-table-columns

% \hskip-4.0cm
\begin{tabular}{c | c c c c c c c c c c c c c c c c | c}
  ID     & 0  & 1  & 2  & 3  & 4 & 5 & 6  & 7  & 8 & 9  & 10  & 11 & 12 & 13 & 14 & 15 & Bytes\\
  \hline
  Array  & 0, & 1, & 2, & 3, &   &   & 6, & 7, &   & 9, & 10, &    &    &    & 14 &    & 9$\times$2 = 18\\
         &    &    &    &    &   &   &    &    &   &    &     &    &    &    &    &    & \\
  Bitmap & 1  & 1  & 1  & 1  & 0 & 0 & 1  & 1  & 0 & 1  &  1  & 0  & 0  & 0  & 1  & 0  & 2\\
         & \omit\span\omit\span\omit\span\omit\mathstrut\upbracefill & & &\omit\span\omit\mathstrut\upbracefill & &\omit\span\omit\mathstrut\upbracefill & & & &\mathstrut\upbracefill & &  \\
  RLE    & \multicolumn{4}{c}{[0, 3]}     &   &   & \multicolumn{2}{c}{[6, 7]} &   & \multicolumn{2}{c}{[9, 10]} &    &    &    & [14, 14]&& 4$\times$2$\times$2 = 16\\
\end{tabular}

%\begin{tabular}{c | c c c c}
%  ID     & 0  & 1  & 2  & 3\\
%  \hline
%  Array  & 0, & 1, & 2, & 3\\
%  Bitmap & 1  & 1  & 1  & 1\\
%\end{tabular}


\end{document}

