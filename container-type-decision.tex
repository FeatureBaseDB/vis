\documentclass{article} % say
\usepackage{tikz}
\usetikzlibrary{patterns}
\renewcommand{\familydefault}{\sfdefault} % for sans font
\begin{document}

% http://cremeronline.com/LaTeX/minimaltikz.pdf

\texttt{Bitmap.Optimize()} uses the values of cardinality (\texttt{c.n} or $N$) and
run count (\texttt{c.runCount()} or $N_R$) to convert container types, achieving
minimal storage size, according to the diagram below.\\

Note that this is not to scale; M is actually much larger than RunMaxSize and
ArrayMaxSize. Also, in the $N$-$N_R$ space, there is no direct correlation between
the size of a region, and the number of sets in the region, or the likelihood of a
random set belonging to it.\\

For example, a random bitmap with a bit density of 5\% is overwhelmingly likely to
be an array, rather than an RLE, \textit{if the set bits are randomly distributed}.\\

\definecolor{pilosa-lightblue}{HTML}{E4EFF4}
\definecolor{pilosa-blue}{HTML}{3C5F8D}
\definecolor{pilosa-green}{HTML}{1DB598}
\definecolor{pilosa-red}{HTML}{ff2B2B}

\begin{tikzpicture}
  % \draw [pattern=north west lines, pattern color=gray] (0, 0) -- (4, 4) -- (8, 4) -- (8, 8) -- (0, 8) -- (0, 0);
  \draw [fill=gray] (0, 0) -- (4, 4) -- (8, 0) -- (8, 8) -- (0, 8) -- (0, 0);
  \fill [pilosa-red] (0, 0) -- (2, 1) -- (7, 1) -- (8, 0) -- (0, 0);
  \fill [pilosa-blue] (2, 1) -- (2, 2) -- (4, 4) -- (7, 1) -- (2, 1);
  \fill [pilosa-green] (0, 0) -- (2, 1) -- (2, 2) -- (0, 0);
  \draw [<->] (0,8.5) node [above] {$N_R$ (runCount)} -- (0,0) node [below left] {0} -- (8.5,0) node [right] {$N$ (cardinality)};
  \draw (0,8) node [left] {$M$};
  \draw (8,0) node [below] {$M$};
  \draw [help lines] (8, 0) -- (8, 8) -- (0, 8);
  \draw [dotted] (0,1) node [left] {RunMaxSize} -- (2,1) -- (2,0) node [below] {ArrayMaxSize};
  % \draw [dotted] (2, 2) -- (2, 9);
  %\draw [solid] (9, 1) -- (2, 1) -- (2, 9);
  %\draw [solid] (0, 0) -- (2, 1);
  %\draw [solid, line width=2] (0, 0) -- (2, 1) -- (8, 1) -- (8, 0) -- (0, 0);
  %\draw [solid, line width=2] (0, 0) -- (2, 2) -- (2, 1);
  %\draw [solid, line width=2] (2, 2) -- (4, 4) -- (8, 4) -- (8, 1);
  \draw [solid] (8, 0) -- (8, 4);
  \draw [dotted] (2, 1) -- (8, 4);
  % \draw [dashed] (0, 0) -- (4, 4);
  % \draw [dashed] (4, 4) -- (9, 9);
  % \draw [dashed] (4, 4) -- (9, 4);
  \draw [dashed] (0, 4) -- (8, 4);
  \draw [dashed] (4, 0) -- (4, 8);
  \node at (1.5, 1) {array};
  \node at (3, .5) {runs};
  \node at (5, 1.5) {bitmap};
  \node at (5.5, 3.5) {impossible};
  \draw (0, 4) node [left] {$\frac{M}{2}$};
  \draw (4, 0) node [below] {$\frac{M}{2}$};
\end{tikzpicture}

$M$ is the maximum number of bits in a container, maxContainerVal+1.\\

Note the impossible region:
\begin{itemize}
  \item The point $(\frac{M}{2}, \frac{M}{2})$ represents a set where all alternating bits are
present, which represents the maximum number of runs for a fixed set size. This is the only possible case for which $N_R = \frac{M}{2}$, and $N_R$ is smaller for all other sets.
  \item The $N = N_R$ diagonal represents sets where every set bit is isolated - all runs are length one. No set can be above this line, as that would mean more runs than set bits.
  \item The $N = M-N_R$ diagonal represents set where every clear bit is isolated - all runs are as close as possible while still being separate. No set can be above this line, as that would mean more runs than clear bits.

\end{itemize}

The remaining region represents all possible sets,
with container type regions separated by the solid lines.

\bigskip

Let $A$ be the array element size in bits, $R$ be the RLE start/end element size. Then array size is $AN$, run size is
$2RN_R$, and bitmap size is $M$.\\

Arrays are smaller than bitmaps when $AN < M$ or $N < \frac{M}{A} = $ ArrayMaxSize.\\
Runs are smaller than bitmaps when $2RN_R < M$ or $N_R < \frac{M}{2R} = $ RunMaxSize.\\
Array are smaller than runs when $AN < 2RN_R$ or $N < 2\frac{R}{A} N_R$.\\

Currently (v0.4.x), $A = R = 32$, $M = 65536$, so\\ 

ArrayMaxSize $ = 2048$.\\
RunMaxSize $ = 1024$.\\
Arrays are smaller than runs when $N < 2 N_R$.\\
\\
\begin{tabular}{l | c | c}
    & 32-bit & 16-bit \\
  \hline
  ArrayMaxSize & 2048 & 4096 \\
  \hline
  RunMaxSize & 1024 & 2048 \\
\end{tabular}

\end{document}

