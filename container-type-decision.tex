\documentclass{article} % say
\usepackage{tikz}
\renewcommand{\familydefault}{\sfdefault} % for sans font
\begin{document}


\texttt{Bitmap.Optimize()} uses the values of cardinality (\texttt{c.n} or $N$) and
run count (\texttt{c.runCount()} or $N_R$) to convert container types, achieving minimal storage size, according to this diagram:\\

\begin{tikzpicture}
  \draw [<->] (0,5) node [left] {$N_R$} -- (0,0) node [below left] {0} -- (5,0) node [below] {$N$};
  \draw [help lines] (5, 0) -- (5, 5) -- (0, 5);
  \draw [dotted] (0,1) node [left] {RunMaxSize} -- (2,1) -- (2,0) node [below] {ArrayMaxSize};
  \draw [solid] (6, 1) -- (2,1) -- (2,6);
  \draw [solid] (0, 0) -- (2, 1);
  \draw [dotted] (2, 1) -- (6, 3);
  \node at (1, 2) {array};
  \node at (3, .5) {runs};
  \node at (3, 3) {bitmap};
\end{tikzpicture}

\bigskip


Let $A$ be the array element size in bits, $M$ be the maximum number of bits in
a bitmap, $R$ be the RLE start/end element size. Then array size is $AN$, run size is
$2RN_R$, and bitmap size is $M$.\\

Arrays are smaller than bitmaps when $AN < M$ or $N < \frac{M}{A} = ArrayMaxSize$.\\
Runs are smaller than bitmaps when $2RN_R < M$ or $N_R < \frac{M}{2R} = RunMaxSize$.\\
Array are smaller than runs when $AN < 2RN_R$ or $N < 2\frac{R}{A} N_R$.\\

Currently (v0.4.x), $A = R = 32$, $M = 65536$, so\\ 

$ArrayMaxSize = 2048$.\\
$RunMaxSize = 1024$.\\
Arrays are smaller than runs when $N < 2 N_R$.\\
\\
\begin{tabular}{l | c | c}
    & 32-bit & 16-bit \\
  \hline
  ArrayMaxSize & 2048 & 4096 \\
  \hline
  RunMaxSize & 1024 & 2048 \\
\end{tabular}

\end{document}

