%\documentclass[tikz]{standalone}
\documentclass{article} % say
\usepackage{tikz}
\usepackage[EULERGREEK]{sansmath}
\usetikzlibrary{arrows,decorations.pathmorphing,backgrounds,positioning,fit,petri}
\usetikzlibrary{shapes.multipart}
\begin{document}

\definecolor{lightblue}{HTML}{E4EFF4}
\definecolor{darkblue}{HTML}{3C5F8D}
\definecolor{darkgreen}{HTML}{1DB598}
\definecolor{boldred}{HTML}{ff2B2B}

% TODO: set text/line color by default or globally
% TODO: thicker lines
% TODO: more specific font options
% https://tex.stackexchange.com/questions/169521/typography-and-style-good-choices-for-font-styles-in-tikz-or-pgfplots-graphics
% TODO: move "size" values to be right-aligned inside their corresponding blocks
% (like travis' svg style)

% a block is one of the elements in the drawing that is surrounded by a rounded rectangle
\newcommand\block[2][]{\tikz\node[draw,thin,rounded
  corners,fill=white,text=darkblue,align=center,minimum width=1mm,#1] {#2};}

\begin{tikzpicture}
  \sffamily
  \sansmath
  \block[] {
    Bitmap storage file\\
    \block[minimum width=16em]{
      Cookie header\\
      % "\hskip 2 em" adds horizontal space to account for the different text widths
      % of "2" and "\frac{}{}"
      % \block[minimum width=45mm]{Magic number \tikz\node[draw,minimum width=1mm,fill=darkblue,text=white]{2};}\\
      \hskip 2em 2 \block[minimum width=12em]{Magic number}\\
      \hskip 2em 2 \block[minimum width=12em,text=darkgreen]{Storage version}\\
      % "\," is a space
      \hskip 2em 4 \block[minimum width=12em]{$\textcolor{red}{N_c}$\textcolor{darkblue}{\, = Container count}}\\
      \hskip 0.3em \textcolor{darkblue}{$\frac{\textcolor{red}{N_c}+7}{8}$} \block[minimum width=12em]{runFlagBitset}
    }\\
    \block[minimum width=16em]{
      Descriptive header\\
      \block[minimum width=15em]{
        8 \block[minimum width=13em]{Key 0}\\
        4 \block[minimum width=13em]{\textcolor{red}{Cardinality 0}}
      }\\
      \vdots \\
      $\times \textcolor{red}{N_c}$
    }\\
    \block[minimum width=16em]{
      Offset header\\
      4 \block[minimum width=14em]{Offset 0}\\
      4 \block[minimum width=14em]{Offset 1}\\
      \vdots \\
      $\times \textcolor{red}{N_c}$
    }\\
    \block[minimum width=16em]{
      Container storage\\
      $A_0$ \block[minimum width=13em]{Container 0}\\
      $A_1$ \block[minimum width=13em]{Container 1}\\
      \vdots \\
      $\times \textcolor{red}{N_c}$
    }\\
    \block[minimum width=16em,text=darkgreen]{
      Operation log\\
      \block[minimum width=15em,text=darkgreen]{
        Operation 0\\
        1 \block[minimum width=13em,text=darkgreen]{Operation type 0}\\
        8 \block[minimum width=13em,text=darkgreen]{Operation value 0}\\
        4 \block[minimum width=13em,text=darkgreen]{Operation checksum 0}
      }\\
      \vdots \\
      $\times N_{ops}$
    }
  };\\

  % TODO: add more horizontal space between the two main blocks

  % containers
  % "at (6,10)" positions this node. without this the positioning is screwy, not
  % sure why. also not sure what the units or origin are - it seems to be
  % relative to the most recently placed node...
  \tikz\node at (6, 10) [thin,rounded corners,fill=white,text=darkblue,align=center,minimum width=1mm] (containers) {
    \quad \block[]{
      Array container\\
      2 \block[minimum width=13em]{Value 0}\\
      2 \block[minimum width=13em]{Value 1}\\
      \vdots\\
      $\times$ \textcolor{red}{cardinality}
    }\\
    \\
    \quad \block[minimum width=50mm]{
      Bitmap container\\
      8 \block[minimum width=13em]{Bitset 0}\\
      8 \block[minimum width=13em]{Bitset 1}\\
      \vdots\\
      $\times$ bitmapN
    }\\
    \\
    \quad \block[minimum width=50mm]{
      RLE container\\
      2 \block[minimum width=13em]{\textcolor{red}{$N_{runs}$}}\\
      4 \block[minimum width=13em]{Run 0}\\
      4 \block[minimum width=13em]{Run 1}\\
      \vdots\\
      $\times$ \textcolor{red}{$N_{runs}$}
    }\\
    \vspace{6cm}
  };

\end{tikzpicture}
\end{document}
