\documentclass[varwidth]{standalone}
\usepackage[table]{xcolor} % for table cell colors
\renewcommand{\familydefault}{\sfdefault} % for sans font
\begin{document}

\definecolor{pilosa-lightblue}{HTML}{E4EFF4}
\definecolor{pilosa-blue}{HTML}{3C5F8D}
\definecolor{pilosa-green}{HTML}{1DB598}
\definecolor{pilosa-red}{HTML}{ff2B2B}

% https://tex.stackexchange.com/questions/62573/using-underbrace-with-table-columns

% reproducing this:
%High-order word:
%0-23                     24-31   32-63
%[iteration              ][seq   ][seed                         ]
%Low-order word:
%0-63
%[id                                                            ]


High-order word:\\
\begin{tabular}{| c | c | c | c | c | c | c | c |}
\hline
0-7  & 8-15  & 16-23  & 24-31  & 32-39 & 40-47 & 48-55  & 56-63 \\
\hline
\omit\span\omit\span\omit\mathstrut\upbracefill & \omit\mathstrut\upbracefill & \omit\span\omit\span\omit\span\omit\mathstrut\upbracefill \\
\multicolumn{3}{c}{iteration} & \multicolumn{1}{c}{seq} & \multicolumn{4}{c}{seed} \\
\end{tabular}\\

Low-order word:\\
\begin{tabular}{| c | c | c | c | c | c | c | c |}
\hline
0-7  & 8-15  & 16-23  & 24-31  & 32-39 & 40-47 & 48-55  & 56-63 \\
\hline
\omit\span\omit\span\omit\span\omit\span\omit\span\omit\span\omit\span\omit\mathstrut\upbracefill \\
\multicolumn{8}{c}{id} \\
\end{tabular}

\end{document}